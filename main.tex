\documentclass{article}
\usepackage[utf8]{inputenc}

\title{Lineær Algebra \\ Projekt X}
\author{Navn Mellemnavn Efternavn - $<$KU-ID@alumni.ku.dk$>$}
\date{\today}

\usepackage{array}      % Matricer
\usepackage{amsmath}    % Generel matematik
\usepackage{inputenc}   % Til danske tegn (Æ, Ø, Å)

% Matematik skrives enten mellem $$ og $$, eller \begin{equation*} og \end{equation*}

% Matricer skrives enten med standard array, eller med pmatrix eller bmatrix for henholdsvis parentes matricer og firkantede parentes matricer (her bruges \begin)

% Udregninger skrives med \begin{align*}...\end{align*}, og bruges sammen med & (alignment) og \\ (linjeskift)

% Med \mathbf skives af og til et fancy U, istedet for bare U, til underrum

% Man kan referere til tidligere eller senere formler, med \ref (\eqref for parenteser) og \label (men så skal \begin{equation*} bruges)

% Numerisk værdi og længde skrives som henholdsvis |x| og ||x||

% For direkte overgivet tekst (der ikke oversættes), bruges \verb for enkelte ord, og \begin{verbatim} for sætninger eller flere ord


\begin{document}

\maketitle

\tableofcontents

\newpage


\section{Introduktion}

% En lille intro-tekst til projektet.

\newpage



\section{Opgave 1}

% Opgave 1 her.
% Generelt, følg mini-skabelonen;

% Jeg har fået følgende opgave ...
% Det første jeg vil gøre er ...
% Dette betyder at jeg nu kan ...
% Repeat de to ovenstående linjer, til et resultat haves
% Dette giver mig resultatet ...

% Hvis der er flere opgaver, så ud-kommenter newpage, og lav flere sections.
%\newpage


%\section{Opgave 2}

% Tekst her

%\newpage


\end{document}
